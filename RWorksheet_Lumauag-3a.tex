% Options for packages loaded elsewhere
\PassOptionsToPackage{unicode}{hyperref}
\PassOptionsToPackage{hyphens}{url}
%
\documentclass[
]{article}
\usepackage{amsmath,amssymb}
\usepackage{iftex}
\ifPDFTeX
  \usepackage[T1]{fontenc}
  \usepackage[utf8]{inputenc}
  \usepackage{textcomp} % provide euro and other symbols
\else % if luatex or xetex
  \usepackage{unicode-math} % this also loads fontspec
  \defaultfontfeatures{Scale=MatchLowercase}
  \defaultfontfeatures[\rmfamily]{Ligatures=TeX,Scale=1}
\fi
\usepackage{lmodern}
\ifPDFTeX\else
  % xetex/luatex font selection
\fi
% Use upquote if available, for straight quotes in verbatim environments
\IfFileExists{upquote.sty}{\usepackage{upquote}}{}
\IfFileExists{microtype.sty}{% use microtype if available
  \usepackage[]{microtype}
  \UseMicrotypeSet[protrusion]{basicmath} % disable protrusion for tt fonts
}{}
\makeatletter
\@ifundefined{KOMAClassName}{% if non-KOMA class
  \IfFileExists{parskip.sty}{%
    \usepackage{parskip}
  }{% else
    \setlength{\parindent}{0pt}
    \setlength{\parskip}{6pt plus 2pt minus 1pt}}
}{% if KOMA class
  \KOMAoptions{parskip=half}}
\makeatother
\usepackage{xcolor}
\usepackage[margin=1in]{geometry}
\usepackage{color}
\usepackage{fancyvrb}
\newcommand{\VerbBar}{|}
\newcommand{\VERB}{\Verb[commandchars=\\\{\}]}
\DefineVerbatimEnvironment{Highlighting}{Verbatim}{commandchars=\\\{\}}
% Add ',fontsize=\small' for more characters per line
\usepackage{framed}
\definecolor{shadecolor}{RGB}{248,248,248}
\newenvironment{Shaded}{\begin{snugshade}}{\end{snugshade}}
\newcommand{\AlertTok}[1]{\textcolor[rgb]{0.94,0.16,0.16}{#1}}
\newcommand{\AnnotationTok}[1]{\textcolor[rgb]{0.56,0.35,0.01}{\textbf{\textit{#1}}}}
\newcommand{\AttributeTok}[1]{\textcolor[rgb]{0.13,0.29,0.53}{#1}}
\newcommand{\BaseNTok}[1]{\textcolor[rgb]{0.00,0.00,0.81}{#1}}
\newcommand{\BuiltInTok}[1]{#1}
\newcommand{\CharTok}[1]{\textcolor[rgb]{0.31,0.60,0.02}{#1}}
\newcommand{\CommentTok}[1]{\textcolor[rgb]{0.56,0.35,0.01}{\textit{#1}}}
\newcommand{\CommentVarTok}[1]{\textcolor[rgb]{0.56,0.35,0.01}{\textbf{\textit{#1}}}}
\newcommand{\ConstantTok}[1]{\textcolor[rgb]{0.56,0.35,0.01}{#1}}
\newcommand{\ControlFlowTok}[1]{\textcolor[rgb]{0.13,0.29,0.53}{\textbf{#1}}}
\newcommand{\DataTypeTok}[1]{\textcolor[rgb]{0.13,0.29,0.53}{#1}}
\newcommand{\DecValTok}[1]{\textcolor[rgb]{0.00,0.00,0.81}{#1}}
\newcommand{\DocumentationTok}[1]{\textcolor[rgb]{0.56,0.35,0.01}{\textbf{\textit{#1}}}}
\newcommand{\ErrorTok}[1]{\textcolor[rgb]{0.64,0.00,0.00}{\textbf{#1}}}
\newcommand{\ExtensionTok}[1]{#1}
\newcommand{\FloatTok}[1]{\textcolor[rgb]{0.00,0.00,0.81}{#1}}
\newcommand{\FunctionTok}[1]{\textcolor[rgb]{0.13,0.29,0.53}{\textbf{#1}}}
\newcommand{\ImportTok}[1]{#1}
\newcommand{\InformationTok}[1]{\textcolor[rgb]{0.56,0.35,0.01}{\textbf{\textit{#1}}}}
\newcommand{\KeywordTok}[1]{\textcolor[rgb]{0.13,0.29,0.53}{\textbf{#1}}}
\newcommand{\NormalTok}[1]{#1}
\newcommand{\OperatorTok}[1]{\textcolor[rgb]{0.81,0.36,0.00}{\textbf{#1}}}
\newcommand{\OtherTok}[1]{\textcolor[rgb]{0.56,0.35,0.01}{#1}}
\newcommand{\PreprocessorTok}[1]{\textcolor[rgb]{0.56,0.35,0.01}{\textit{#1}}}
\newcommand{\RegionMarkerTok}[1]{#1}
\newcommand{\SpecialCharTok}[1]{\textcolor[rgb]{0.81,0.36,0.00}{\textbf{#1}}}
\newcommand{\SpecialStringTok}[1]{\textcolor[rgb]{0.31,0.60,0.02}{#1}}
\newcommand{\StringTok}[1]{\textcolor[rgb]{0.31,0.60,0.02}{#1}}
\newcommand{\VariableTok}[1]{\textcolor[rgb]{0.00,0.00,0.00}{#1}}
\newcommand{\VerbatimStringTok}[1]{\textcolor[rgb]{0.31,0.60,0.02}{#1}}
\newcommand{\WarningTok}[1]{\textcolor[rgb]{0.56,0.35,0.01}{\textbf{\textit{#1}}}}
\usepackage{graphicx}
\makeatletter
\def\maxwidth{\ifdim\Gin@nat@width>\linewidth\linewidth\else\Gin@nat@width\fi}
\def\maxheight{\ifdim\Gin@nat@height>\textheight\textheight\else\Gin@nat@height\fi}
\makeatother
% Scale images if necessary, so that they will not overflow the page
% margins by default, and it is still possible to overwrite the defaults
% using explicit options in \includegraphics[width, height, ...]{}
\setkeys{Gin}{width=\maxwidth,height=\maxheight,keepaspectratio}
% Set default figure placement to htbp
\makeatletter
\def\fps@figure{htbp}
\makeatother
\setlength{\emergencystretch}{3em} % prevent overfull lines
\providecommand{\tightlist}{%
  \setlength{\itemsep}{0pt}\setlength{\parskip}{0pt}}
\setcounter{secnumdepth}{-\maxdimen} % remove section numbering
\ifLuaTeX
  \usepackage{selnolig}  % disable illegal ligatures
\fi
\usepackage{bookmark}
\IfFileExists{xurl.sty}{\usepackage{xurl}}{} % add URL line breaks if available
\urlstyle{same}
\hypersetup{
  pdftitle={RWorksheet\_Lumauag\#3a},
  pdfauthor={Matt Andrei Lumauag},
  hidelinks,
  pdfcreator={LaTeX via pandoc}}

\title{RWorksheet\_Lumauag\#3a}
\author{Matt Andrei Lumauag}
\date{2024-09-30}

\begin{document}
\maketitle

\begin{Shaded}
\begin{Highlighting}[]
\CommentTok{\#1}
\CommentTok{\#a}
\NormalTok{first11Letter }\OtherTok{\textless{}{-}}\NormalTok{ letters[}\DecValTok{1}\SpecialCharTok{:}\DecValTok{11}\NormalTok{]}
\NormalTok{first11Letter}
\end{Highlighting}
\end{Shaded}

\begin{verbatim}
##  [1] "a" "b" "c" "d" "e" "f" "g" "h" "i" "j" "k"
\end{verbatim}

\begin{Shaded}
\begin{Highlighting}[]
\CommentTok{\#b}
\NormalTok{oddNumberedList }\OtherTok{\textless{}{-}}\NormalTok{ LETTERS[}\FunctionTok{seq}\NormalTok{(}\DecValTok{1}\NormalTok{, }\DecValTok{26}\NormalTok{, }\DecValTok{2}\NormalTok{)]}
\NormalTok{oddNumberedList}
\end{Highlighting}
\end{Shaded}

\begin{verbatim}
##  [1] "A" "C" "E" "G" "I" "K" "M" "O" "Q" "S" "U" "W" "Y"
\end{verbatim}

\begin{Shaded}
\begin{Highlighting}[]
\CommentTok{\#c    }
\NormalTok{vowelLetters }\OtherTok{\textless{}{-}}\NormalTok{ LETTERS[}\FunctionTok{c}\NormalTok{(}\DecValTok{1}\NormalTok{, }\DecValTok{5}\NormalTok{, }\DecValTok{9}\NormalTok{,  }\DecValTok{15}\NormalTok{, }\DecValTok{21}\NormalTok{)]}
\NormalTok{vowelLetters}
\end{Highlighting}
\end{Shaded}

\begin{verbatim}
## [1] "A" "E" "I" "O" "U"
\end{verbatim}

\begin{Shaded}
\begin{Highlighting}[]
\CommentTok{\#d}
\NormalTok{LastLetter }\OtherTok{\textless{}{-}}\NormalTok{ letters[}\DecValTok{22}\SpecialCharTok{:}\DecValTok{26}\NormalTok{]}
\NormalTok{LastLetter}
\end{Highlighting}
\end{Shaded}

\begin{verbatim}
## [1] "v" "w" "x" "y" "z"
\end{verbatim}

\begin{Shaded}
\begin{Highlighting}[]
\CommentTok{\#2}

\CommentTok{\#a}
\NormalTok{city }\OtherTok{\textless{}{-}} \FunctionTok{c}\NormalTok{(}\StringTok{"Tugegarao City"}\NormalTok{, }\StringTok{"Manila"}\NormalTok{, }\StringTok{"Iloilo City"}\NormalTok{, }\StringTok{"Tacloban"}\NormalTok{, }\StringTok{"Samal Island"}\NormalTok{, }\StringTok{"Davao City"}\NormalTok{)}
\CommentTok{\#b}
\NormalTok{temp }\OtherTok{\textless{}{-}} \FunctionTok{c}\NormalTok{(}\DecValTok{42}\NormalTok{, }\DecValTok{39}\NormalTok{, }\DecValTok{34}\NormalTok{, }\DecValTok{34}\NormalTok{, }\DecValTok{30}\NormalTok{, }\DecValTok{27}\NormalTok{)}

\CommentTok{\#c}
\NormalTok{CityTemp }\OtherTok{\textless{}{-}} \FunctionTok{data.frame}\NormalTok{(city, temp)}

\CommentTok{\#d}
\FunctionTok{names}\NormalTok{(CityTemp) }\OtherTok{\textless{}{-}} \FunctionTok{c}\NormalTok{(}\StringTok{"City"}\NormalTok{, }\StringTok{"Temperature"}\NormalTok{)}
\NormalTok{CityTemp}
\end{Highlighting}
\end{Shaded}

\begin{verbatim}
##             City Temperature
## 1 Tugegarao City          42
## 2         Manila          39
## 3    Iloilo City          34
## 4       Tacloban          34
## 5   Samal Island          30
## 6     Davao City          27
\end{verbatim}

\begin{Shaded}
\begin{Highlighting}[]
\CommentTok{\#e}
\FunctionTok{str}\NormalTok{(CityTemp)}
\end{Highlighting}
\end{Shaded}

\begin{verbatim}
## 'data.frame':    6 obs. of  2 variables:
##  $ City       : chr  "Tugegarao City" "Manila" "Iloilo City" "Tacloban" ...
##  $ Temperature: num  42 39 34 34 30 27
\end{verbatim}

\begin{Shaded}
\begin{Highlighting}[]
\CommentTok{\#f}
\NormalTok{CityTemp[}\DecValTok{3}\SpecialCharTok{:}\DecValTok{4}\NormalTok{, ]}
\end{Highlighting}
\end{Shaded}

\begin{verbatim}
##          City Temperature
## 3 Iloilo City          34
## 4    Tacloban          34
\end{verbatim}

\begin{Shaded}
\begin{Highlighting}[]
\CommentTok{\#g}
\NormalTok{CityTemp[}\FunctionTok{which.max}\NormalTok{(CityTemp}\SpecialCharTok{$}\NormalTok{Temperature), ]}
\end{Highlighting}
\end{Shaded}

\begin{verbatim}
##             City Temperature
## 1 Tugegarao City          42
\end{verbatim}

\begin{Shaded}
\begin{Highlighting}[]
\NormalTok{CityTemp[}\FunctionTok{which.min}\NormalTok{(CityTemp}\SpecialCharTok{$}\NormalTok{Temperature), ]}
\end{Highlighting}
\end{Shaded}

\begin{verbatim}
##         City Temperature
## 6 Davao City          27
\end{verbatim}

\begin{Shaded}
\begin{Highlighting}[]
\CommentTok{\#2}
\CommentTok{\#a}
\NormalTok{matrx }\OtherTok{\textless{}{-}} \FunctionTok{matrix}\NormalTok{(}\FunctionTok{c}\NormalTok{(}\DecValTok{1}\NormalTok{,}\DecValTok{2}\NormalTok{,}\DecValTok{3}\NormalTok{,}\DecValTok{4}\NormalTok{,}\DecValTok{5}\NormalTok{,}\DecValTok{6}\NormalTok{,}\DecValTok{7}\NormalTok{,}\DecValTok{8}\NormalTok{,}\DecValTok{11}\NormalTok{,}\DecValTok{12}\NormalTok{,}\DecValTok{13}\NormalTok{,}\DecValTok{14}\NormalTok{), }\AttributeTok{nrow =}  \DecValTok{3}\NormalTok{, }\AttributeTok{ncol =} \DecValTok{4}\NormalTok{)}
\NormalTok{matrx}
\end{Highlighting}
\end{Shaded}

\begin{verbatim}
##      [,1] [,2] [,3] [,4]
## [1,]    1    4    7   12
## [2,]    2    5    8   13
## [3,]    3    6   11   14
\end{verbatim}

\begin{Shaded}
\begin{Highlighting}[]
\CommentTok{\#b}
\NormalTok{matrx2 }\OtherTok{\textless{}{-}}\NormalTok{ matrx}\SpecialCharTok{*}\DecValTok{2}
\NormalTok{matrx2}
\end{Highlighting}
\end{Shaded}

\begin{verbatim}
##      [,1] [,2] [,3] [,4]
## [1,]    2    8   14   24
## [2,]    4   10   16   26
## [3,]    6   12   22   28
\end{verbatim}

\begin{Shaded}
\begin{Highlighting}[]
\CommentTok{\#c}
\NormalTok{matrx[}\DecValTok{2}\NormalTok{, ]}
\end{Highlighting}
\end{Shaded}

\begin{verbatim}
## [1]  2  5  8 13
\end{verbatim}

\begin{Shaded}
\begin{Highlighting}[]
\CommentTok{\#d}
\NormalTok{matrx[}\DecValTok{1}\SpecialCharTok{:}\DecValTok{2}\NormalTok{, }\DecValTok{3}\SpecialCharTok{:}\DecValTok{4}\NormalTok{]}
\end{Highlighting}
\end{Shaded}

\begin{verbatim}
##      [,1] [,2]
## [1,]    7   12
## [2,]    8   13
\end{verbatim}

\begin{Shaded}
\begin{Highlighting}[]
\CommentTok{\#e}
\NormalTok{matrx[}\DecValTok{3}\NormalTok{, }\DecValTok{2}\SpecialCharTok{:}\DecValTok{3}\NormalTok{]}
\end{Highlighting}
\end{Shaded}

\begin{verbatim}
## [1]  6 11
\end{verbatim}

\begin{Shaded}
\begin{Highlighting}[]
\CommentTok{\#f}
\NormalTok{matrx[ ,}\DecValTok{4}\NormalTok{]}
\end{Highlighting}
\end{Shaded}

\begin{verbatim}
## [1] 12 13 14
\end{verbatim}

\begin{Shaded}
\begin{Highlighting}[]
\CommentTok{\#g}
\FunctionTok{rownames}\NormalTok{(matrx) }\OtherTok{\textless{}{-}} \FunctionTok{c}\NormalTok{(}\StringTok{"isa"}\NormalTok{, }\StringTok{"dalawa"}\NormalTok{, }\StringTok{"tatlo"}\NormalTok{)}
\FunctionTok{colnames}\NormalTok{(matrx) }\OtherTok{\textless{}{-}} \FunctionTok{c}\NormalTok{(}\StringTok{"uno"}\NormalTok{, }\StringTok{"dos"}\NormalTok{, }\StringTok{"tres"}\NormalTok{, }\StringTok{"quatro"}\NormalTok{)}
\NormalTok{matrx}
\end{Highlighting}
\end{Shaded}

\begin{verbatim}
##        uno dos tres quatro
## isa      1   4    7     12
## dalawa   2   5    8     13
## tatlo    3   6   11     14
\end{verbatim}

\begin{Shaded}
\begin{Highlighting}[]
\CommentTok{\#h}
\FunctionTok{dim}\NormalTok{(matrx) }\OtherTok{\textless{}{-}} \FunctionTok{c}\NormalTok{(}\DecValTok{6}\NormalTok{, }\DecValTok{2}\NormalTok{)}
\NormalTok{matrx}
\end{Highlighting}
\end{Shaded}

\begin{verbatim}
##      [,1] [,2]
## [1,]    1    7
## [2,]    2    8
## [3,]    3   11
## [4,]    4   12
## [5,]    5   13
## [6,]    6   14
\end{verbatim}

\end{document}
